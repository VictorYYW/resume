%%%%%%%%%%%%%%%%%%%%%%%%%%%%%%%%%%%%%%%%%
% Medium Length Professional CV
% LaTeX Template
% Version 2.0 (8/5/13)
%
% This template has been downloaded from:
% http://www.LaTeXTemplates.com
%
% Original author:
% Trey Hunner (http://www.treyhunner.com/)
%
% Important note:
% This template requires the resume.cls file to be in the same directory as the
% .tex file. The resume.cls file provides the resume style used for structuring the
% document.
%
%%%%%%%%%%%%%%%%%%%%%%%%%%%%%%%%%%%%%%%%%

%----------------------------------------------------------------------------------------
%	PACKAGES AND OTHER DOCUMENT CONFIGURATIONS
%----------------------------------------------------------------------------------------

\documentclass{resume} % Use the custom resume.cls style

\usepackage[left=0.75in,top=0.7in,right=0.75in,bottom=0.2in]{geometry} % Document margins
\usepackage{biblatex}
\usepackage{hyperref}


\name{Yongwei Yuan} % Your name
\address{\href{mailto:yuan311@purdue.edu}{yuan311@purdue.edu}} % Your address
\address{\url{https://victoryyw.github.io/}}
\address{+1 734-881-4100}

\begin{document}
%----------HEADING-----------------
% \begin{tabular*}{\textwidth}{l@{\extracolsep{\fill}}r}
%   \textbf{}{\huge \bf{Yongwei Yuan}} & Email : {yuan311@purdue.edu}\\
%   {Ph.D. student in computer science} & {Mobile : +1-734-881-4100} \\
% % {UMID : 26211890} & {} \\
% \end{tabular*}
%\begin{document}
%----------------------------------------------------------------------------------------
%	EDUCATION SECTION
%----------------------------------------------------------------------------------------

\begin{rSection}{\large Education}
  \begin{rSubsection*}
    {Purdue University, West Lafayette}{Indiana, United States}
    {Ph.D. in Computer Science}{\textit{Aug. 2020 - now}}
  \item CS580: Algorithm Design, Analysis, And Implementation
  \item CS560: Reasoning About Programs
  \end{rSubsection*}

  \begin{rSubsection*}
    {University of Michigan, Ann Arbor}{Michigan, United States}
    {B.S. in Computer Science}{\textit{Sept. 2018 - May. 2020}}
  \end{rSubsection*}

  \begin{rSubsection*}{Shanghai Jiao Tong University}{Shanghai, China}
  {B.S. in Electrical and Computer Engineering}{\textit {Sept. 2016 - Aug. 2020}}
  \end{rSubsection*}

\end{rSection}

%----------------------------------------------------------------------------------------
%   RESEARCH SECTION
%----------------------------------------------------------------------------------------
\begin{rSection}{\large Research}
 
  \begin{rSubsection}
    {SIS-Lambda}{Advisor: Prof. Roopsha Samanta}
    {Department of Computer Science, Purdue University}{Aug. 2020 - now}
  \item To take advantage of relational properties and add semantic bias to
    exmaple-based synthesis of functional programs
  \item Found evidence that augmenting user-provided examples could resolve the
    ambiguity problem in synthesis
  \item Working on the formalism of example augmentation process through defining a
    refinement type system
  \end{rSubsection}
  \begin{rSubsection}
    {Pattern Matching with Typed Holes}{Advisor: Prof. Cyrus Omar}
    {Computer Science and Engineering, University of Michigan}{Jan. 2020 - now}
  \item To support incomplete structural pattern matching and provide feedback
    to users in every possible editor state
  \item Developed a simply typed lambda calculus, Peanut, where reasoning about exhaustiveness
    and redundancy is mapped to the problem of deriving first-order entailment
    between constraints
  \item Developed an operational semantics that allows us to evaluate match
    expressions in the presence of holes
  \item Proved key metatheoretic properties and formalized a procedure capable of
    deciding the necessary entailment
  \end{rSubsection}

\end{rSection}
%\vspace{0.5em}

%------------------------------------------------
%----------------------------------------------------------------------------------------
%	RESEARCH EXPERIENCE SECTION
%----------------------------------------------------------------------------------------

\begin{rSection}{\large Projects}
%------------------------------------------------
\begin{rSubsection}
  {Variable Usability in Hazel}{Advisor: Prof. Cyrus Omar}
  {Computer Science and Engineering, University of Michigan}{Sept. 2019 - Mar. 2020}
\item To improve the usability of variables in \href{https://hazel.org/build/dev/index.html}{\textsf{Hazel}}, a live functional programming environment featuring typed holes
\item Added support for static variable usage analysis for both the one under the
  cursor and other variables
\item Provided variable-based program navigation 
\item Designed and Implemented secondary notations to display variable usage information
\item Exposed hidden dependencies between the binding site and the usage site of
  variables to programmers
\end{rSubsection}
\begin{rSubsection}
  {Bugbase V2}{Prof. Baris Kasikci}
  {Computer Science and Engineering, University of Michigan}{Apr. 2019 - Sept. 2019}
\item To verify the effectiveness of existing symbolic-execution-based bug-finding tools, like KLEE
\item Reproduced dozens of bugs in docker containers, involving hacking into the codebase and transforming LLVM IR when necessary
\item Digged into the codebase of large-scale software systems to analyze root
  cause for multiple bugs
\end{rSubsection}

\begin{rSubsection}{Review-Me Automation}{Advisor: Prof. Tawanna Dillahunt}
  {School of Information, University of Michigan}{Jun. 2019 - Oct. 2019}
\item Provided back-end support for \href{https://review-me.us}{review-me}, a system dedicated to provide expert resume feedback for job seekers
\item Automated the process of reviewing resumes and providing feedback by taking advantage of crowdsourcing
\end{rSubsection}

\end{rSection}
%------------------------------------------------ %\vspace{0.5em}  
\begin{rSection}{\large Awards}
Honorable Mention, Mathematical Contest in Modeling \hfill {Apr. 2017}

\end{rSection}

%----------------------------------------------------------------------------------------
%	TECHNICAL STRENGTHS SECTION
%----------------------------------------------------------------------------------------

% \begin{rSection}{\large Skills}
% \begin{tabular}{ @{} >{\bfseries}l @{\hspace{6ex}} l }
% Programming Languages & ReasonML, OCaml, C/C++, Python, Javascript 
% \end{tabular}
% \end{rSection}

%\begin{rSection}{\large Publications\&Patents}
% Meng Qu$^*$, {\bf Chun Ni}$^*$, Mufan Chen$^*$ \emph{et al}., "Automatic  diabetic  retinopathy  diagnosis  using adjustable  ophthalmoscope  and  multi-scale  line  operator". \underline{\emph{Pervasive and Mobile Computing}}  ($^*$Co-first author)\\
% Ying Cui, Wen He, {\bf Chun Ni}, Chengjun Guo, and Zhi Liu. "Energy-efficient resource allocation for cache-assisted mobile edge computing". \underline{\emph{IEEE LCN}} \emph{2017}.  \\
% Bin Sheng, Ling Dai, {\bf Chun Ni}. A Method for Automatically Detecting Micro Aneurysms Based on Multisieving Convolution Neural Network. Patent Pending. 
%\end{rSection}


%%\begin{rSection}{Leadership}
%%\begin{rSubsection}{Youth Volunteer Team $|$ Director}{}{}

%%\item Organized charity sale for left-behind children. Collected idle stuff and sold them in the campus, finally got 2648RMB.
%%\item Established and organized a long term volunteer service to assist in rehabilitation of autistic children 
%%\end{rSubsection}
%%\begin{rSubsection}{Debate Team of School $|$ Coach}{}{}

%%\item Championship member of University Debate Race, Best Debater

%%\end{rSubsection}
%%\end{rSection}
%------------------------------------------------


%----------------------------------------------------------------------------------------

\end{document}
